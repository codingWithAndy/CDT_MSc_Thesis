% !TEX root = ../thesis.tex
\chapter{Introduction}
	\label{chap:intro}



	\section{Motivations}
		\label{sec:intro_motivation} 
	For the prior eight years, we have had involvement in some form of an educational environment. Seven of these years involve being a teacher within secondary and sixth form schools. While the focus of teaching is perceived to create lessons for students to learn and grow, we found more and more as the years went on that this wasn't the case. The focus was actually on providing reports about the students, which required data about the students from formal assessments. While having assessments to gauge the level that a student is at is an essential part of education. However, creating, marking, analysing and providing feedback for 30 students or more per class is a time-consuming task. Therefore, this assessment practice takes away the educators' time to do what is essential, creating meaningful lessons tailored for the students.
	
	Therefore, our motivation is to create a tool for educators that will empower them to allow technology to do what it is good at and focus on what they are good at, creating and delivering lessons. To shape future generations views.


	\subsection{Objective}
		\label{sec:intro_objective} 



	\section{Overview}  
		\label{sec:intro_overview} 



	\section{Contributions} 
		\label{sec:intro_contribs} 

The main contributions of this work can be seen as follows:

		\begin{description}	

			\item[\(\bullet\) A \LaTeX{} thesis template]\hfill

Modify this document by adding additional \TeX{} files for your top level content chapters. 

			\item[\(\bullet\) A typesetting guide for useful primitive elements]\hfill

Use the building blocks within this template to typeset each part of your document.
Aim to use simple and reusable elements to keep your \LaTeX{} code neat and to make your document consistently styled throughout.

			\item[\(\bullet\) A review of how to find and cite external resources]\hfill

We review techniques and resources for finding and properly citing resources from the prior academic literature and from online resources.

		\end{description}