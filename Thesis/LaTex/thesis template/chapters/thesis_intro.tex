% !TEX root = ../thesis.tex
\chapter{Introduction}
	\label{chap:intro}
	We have set out to create a tool that can simulate a small scale comparative judgement experiment on what users think about tweets getting compared against each other. This experiment is in light of our stakeholder getting commissioned by the Welsh government to implement a comparative judgement system nationally for all schools in Wales. Comparative judgement is a technique that has been around for almost 100 years. However, while the process can improve results and reduce cognitive loads for teachers and markers, especially at the scale that the stakeholder's implementation will have to work at, it can still require a considerable number of combinations to be marked and compared. For this experiment, we decided to use tweets based on Brexit.
	
	Therefore, we have created a tool that allows users to see a sub-sample of the combinations. Once the users have viewed the varieties, an overall ranking of the results will get created. Two methods got implemented, a more traditional comparative judgement method and an Elo style ranking.
	
	We then aimed to use NLP techniques to see any insights we could find within the tweets. We intended to extract information on the tweets to see if we could find patterns that would give us insights into what might have impacted the tweets final scores.
	
	The study got broken up into two parts. Part one was a web app to gather user's views on the tweets, and the second part was exploring NLP techniques within a Jupyter Notebook.


	\section{Motivations}
		\label{sec:intro_motivation} 
	For the prior eight years, we have had involvement in some form of an educational environment. Seven of these years involve being a teacher within secondary and sixth form schools. While the focus of teaching is perceived to create lessons for students to learn and grow, we found more and more as the years went on that this wasn't the case. The focus was actually on providing reports about the students, which required data about the students from formal assessments. While having assessments to gauge the level that a student is at is an essential part of education. However, creating, marking, analysing and providing feedback for 30 students or more per class is a time-consuming task. Therefore, this assessment practice takes away the educators' time to do what is essential, creating meaningful lessons tailored for the students.
	
	Therefore, our motivation is to create a tool for educators that will empower them to allow technology to do what it is good at and focus on what they are good at, creating and delivering lessons. To shape future generations views.


	\subsection{Objective}
		\label{sec:intro_objective} 



	\section{Overview}  
		\label{sec:intro_overview} 



	\section{Contributions} 
		\label{sec:intro_contribs} 

The main contributions of this work can be seen as follows:

		\begin{description}	

			\item[\(\bullet\) A \LaTeX{} thesis template]\hfill

Modify this document by adding additional \TeX{} files for your top level content chapters. 

			\item[\(\bullet\) A typesetting guide for useful primitive elements]\hfill

Use the building blocks within this template to typeset each part of your document.
Aim to use simple and reusable elements to keep your \LaTeX{} code neat and to make your document consistently styled throughout.

			\item[\(\bullet\) A review of how to find and cite external resources]\hfill

We review techniques and resources for finding and properly citing resources from the prior academic literature and from online resources.

		\end{description}