% !TEX root = ../thesis.tex
\chapter{Typesetting your thesis}
	\label{chap:typesetting}
	
This document is intended as both a \LaTeX{} thesis template and as a brief tutorial on structuring and typesetting your thesis in the \LaTeX{} programming language.
It is by no means a comprehensive \LaTeX{} guide, but fortunately there are plenty of those available.
	
The following are some powerful online resources for learning about \LaTeX{}:

	\begin{description}	

		\item[\(\bullet\) Overleaf Documentation for \LaTeX{}]\hfill

Overleaf \cite{overleafdocs} is an online browser-based \LaTeX{} IDE which stores your document in the cloud and provides live recompilation as you type.
The documentation on Overleaf's website has a good knowledge base of examples for how to typeset things cleanly and simply in \LaTeX{} code. 

		\noindent See: {\small \url{https://www.overleaf.com/learn}}

		\item[\(\bullet\) \TeX{} StackExchange, the StackOverflow site dedicated to \TeX{} questions]\hfill

\TeX{} StackExchange \cite{texstackexchange} is sub-community of the StackOverflow network dedicated to questions about the \TeX{} family of typesetting tools including \LaTeX{}, Bib\TeX{} and others.
A vast majority of the time it is unlikely that the question or issue you are facing is one that has not been encountered before, and this site is more than likely to be able to point you in the correct direction. 
		
		\noindent See: {\small \url{https://tex.stackexchange.com}}
		
	\end{description}

	
	\section{Referencing items within this document}

In \cref{sec:resources_bibtex} we saw examples of how to typeset citations for resources we had stored in an external Bib\TeX{} file.
However, often we would like to accurately refer to the location of a resource or region of text stored somewhere else within this document\footnote{Like at the beginning of the last sentence, when we referred to section \cref{sec:resources_bibtex}.}.
To do this we need to annotate our \LaTeX{} code with \lstinline|\label{key}| statements which will take on the numeric (or otherwise formatted) identifier for the current chapter, section, figure, table, equation, listing \etc where they are directly defined.
To insert an inline reference to the label alone you can use the \lstinline|\ref{key}| command which works similarly to the \lstinline|\cite{key}| used for external references.
If you would like to automatically include and format the name of the reference target, then you can use \lstinline|\cref{key}|.
In the event we choose to reorder or add additional content to the document (\ie an action that would change chapter/section numbering), the document will still compile to a PDF with the correct references inserted for each \lstinline|\cref{key}| command.


	\input{./chapters/thesis_typesetting_equations}
		
	% !TEX root = ../thesis.tex
\section{Figures}
	\label{sec:typesetting_figures}

In this template figures are numbered starting with the current chapter number followed by a figure number that resets to 1 each new chapter.
As you can see below, the first figure is labelled \cref{fig:dragon} because we are in \cref{chap:typesetting}.
	
Figures in \LaTeX{} are defined using a \lstinline|\begin{figure}...\end{figure}| environment and often immediately begin rendering in centre aligned mode by calling \lstinline|\centering|.
\Cref{lst:latex_figure} below shows the \LaTeX{} code used to typeset \cref{fig:dragon}. \Cref{fig:example_2x1,fig:example_2x2} are defined similarly and make additional use of the \lstinline|\subfloat| command to position multiple images within a single figure environment, each with their own automatically incremented labels and individual captions.

	\lstinputlisting[label={lst:latex_figure}, caption={An example \LaTeX{} excerpt demonstrating how to typeset \cref{fig:dragon} with a simple caption.}]{./listings/example_figure.tex}


	\subsection{Consistent presentation throughout the document}

Figures work best in a document when you use a consistent style for formatting and captioning them, and make sure that figures always actively support the content of the main text. 


	\subsection{Justified use of space in the document}

All figures must be referred to directly in the main text of the document and discussed with meaningful and in depth critical analysis.
If you don't need to use the figure to leverage and support your discussion then it is just taking up space and padding out the document.
For example, you can use a command like \lstinline|\cref{fig:dragon}| to automatically get the figure label for \cref{fig:dragon}. 	

		% !TEX root = ../thesis.tex
% [H] means put the figure HERE, directly when you input this code.
% Remove this to let LaTeX place the figure where it decides is best
\begin{figure}
	\centering

	% We set the width of the figure based on the width of one line 
	% of text on the page. The value can be tuned to any value in 
	% [0.0, 1.0] to scale the image while maintaining its aspect ratio.
	\includegraphics[width=1.0\textwidth]{dragon.jpg}
	
	% Caption is defined with a short and long version. The short 
	% version is shown in the List of Figures section, and the long 
	% version is used directly with the figure. 		
	\caption[Short caption.]{Long caption and citation \cite{whittle15_dragons}.
	
	% Figure labels should be defined at the end of the caption to
	% ensure proper numbering.
	\label{fig:dragon}
	}
\end{figure}


	\subsection{Placement that supports and enhances the flow of the document}

All figures shown in your document should be displayed in relevant locations.
The definition of ``relevant'' can be a matter of preference, however.
\LaTeX{} defaults to balancing figure and text placement, often preferring, \eg the top or bottom of a nearby page, rather than the location at which you insert a figure command.
This style is often preferable, especially for two-column documents, but can sometimes break up the flow of a single column dissertation-like document, for which it is can sometimes be better to reference figures just after they have been alluded to in the main text.
The examples in this document control figure placement by forcing them to appear exactly where referenced using the \lstinline|[H]| parameter.
See the example figures and online resources for more information about figure placement options.


	\subsection{Avoid directly importing other peoples images}

You should avoid using other people's figures whenever possible, and instead create your own figures for visualising the specific methods and data you are working with in a way directly relevant to your project. 


	\subsection{Format sub-figures in \LaTeX{}, not in the image itself}

Construct sub-figures from multiple image files in \LaTeX{} not in the image file itself.
This allows you to tweak the positioning and layout without having to modify the images.
It also allows for automatic formatting and numbering of captions and sub-captions. \Cref{fig:example_2x1,fig:example_2x2} show examples of side-by-side and quad layouts respectively.
		
		% !TEX root = ../thesis.tex
% [H] means put the figure HERE, directly when you input this code.
% Remove this to let LaTeX place the figure where it decides is best
\begin{figure}[H]
	\centering
	
% We use a figure width of 48.5% of the width of one line of text on 
% the page so there is some space between the images.
	\subfloat[Left image sub-caption.]{
		\includegraphics[width=0.485\textwidth]{dragon.jpg}\label{fig:example_2x1_a}
	}\hfill % Spacing between sub-figures displayed next to each other.
	\subfloat[Right image sub-caption.]{
		\includegraphics[width=0.485\textwidth]{dragon.jpg}\label{fig:example_2x1_b}
	}\\ % New line before caption.

% Caption is defined with a short and long version. The short version is shown in the 
% List of Figures section, and the long version is used directly with the figure. 	
	\caption[A demonstration of a 2x1 sub-figure layout.]{
Construct sub-figures from multiple image files in \LaTeX{} rather than in the image file itself.
This allows you to tweak the positioning and layout without having to modify the images.
It also allows for automatic formatting and numbering of captions and sub-captions.
Image of glass dragons rendered using Path Tracing \cite{whittle15_dragons}.
	
% Figure labels should be defined at the end of the caption to ensure proper numbering.
	\label{fig:example_2x1}
	}
	
\end{figure}
		
		% !TEX root = ../thesis.tex
% [H] means put the figure HERE, directly when you input this code.
% Remove this to let LaTeX place the figure where it decides is best
\begin{figure}[H]
	\centering
	
% We use a figure width of 48.5% of the width of one line of text on 
% the page so there is some space between the images.
	\subfloat[Top-Left image sub-caption.]{
		\includegraphics[width=0.485\textwidth]{dragon.jpg}\label{fig:example_2x2_a}
	}\hfill % Spacing between sub-figures displayed next to each other.
	\subfloat[Top-Right image sub-caption.]{
		\includegraphics[width=0.485\textwidth]{dragon.jpg}\label{fig:example_2x2_b}
	}\\ % New line before caption.
	\subfloat[Bottom-Left image sub-caption.]{
		\includegraphics[width=0.485\textwidth]{dragon.jpg}\label{fig:example_2x2_c}
	}\hfill % Spacing between sub-figures displayed next to each other.
	\subfloat[Bottom-Right image sub-caption.]{
		\includegraphics[width=0.485\textwidth]{dragon.jpg}\label{fig:example_2x2_d}
	}\\ % New line before caption.
		
% Caption is defined with a short and long version. The short version is shown in the 
% List of Figures section, and the long version is used directly with the figure. 	
	\caption[A demonstration of a 2x2 sub-figure layout.]{
A demonstration of a 2x2 sub-figure layout.
Between (a)--(b) and (c)--(d) we use \texttt{\(\backslash\)hfill} and between (b)--(c) we use a new line.
Image of glass dragons rendered using Path Tracing \cite{whittle15_dragons}.
	
% Figure labels should be defined at the end of the caption to ensure proper numbering.
	\label{fig:example_2x2}
	}
	
\end{figure}


	\subsection{Robust captions that can stand in isolation}

Figures need to be captioned such that they can be viewed in isolation and still be meaningful to the viewer.
There will likely be some duplication of information that is written in the main text, but this is intended. 


	\subsection{Proper attribution and citation of images}
		\label{sec:typesetting_figures_citation}
		
If an image does not belong to you it \textbf{must} be cited directly in the figure caption. \textbf{It is not correct to put a URL in the figure caption directly.}
A URL in isolation is not an accurate or reliable way of directing a future reader to the exact content you are referencing.
Instead, make a new entry in your \lstinline|citations.bib| file and then reference that citation in the caption using the \lstinline|\cite{key}| command.
\Cref{fig:dragon,fig:example_2x1,fig:example_2x2} each include a statement in the caption stating ``Image of glass dragons rendered using Path Tracing \cite{whittle15_dragons}.''.
When adding the Bib\TeX{} entry, try to find the proper information about the original author and source document to strengthen the citation in case the URL changes.

	
	\input{./chapters/thesis_typesetting_listings}
	
	% !TEX root = ../thesis.tex
\section{Tables}
	\label{sec:typesetting_tables}
	
Tables are also quite predictably captioned and formatted the same way.
It is important to decide on a style for how you will organise your data and apply that style consistently for all of your tables.
\Cref{tbl:example_table} shows one possible way of styling your data but is by no means the only way of doing so neatly.
Consistency is the key. 
	
	% !TEX root = ../thesis.tex
% It's often a good idea to generate the LaTeX code for tables (python script or similar) 
% so that if you rerun your code you don't have to typeset your results again by hand!
\begin{table}[H]
	\centering
	\small
	\caption[A demonstration of a table typeset in \LaTeX{}.]{An example of a table formatted with a caption.}
	\begin{tabular}{c|c|c|c|c|c}
		\toprule
		Some & Relevant & Fields & From & Your & Data\\
		\midrule
		0 & 0 & 0 & 0 & 0 & 0\\
		1 & 1 & 1 & 1 & 1 & 1\\
		2 & 2 & 2 & 2 & 2 & 2\\
		\bottomrule
	\end{tabular}
	\label{tbl:example_table}
\end{table}

% This alternative version helps you fit large tables into smaller spaces by scalling 
% (e.g., 0.8 and 0.46\textwidth). It is normally better to format your data in a way that fits 
% in the standard space, however.
%
%\begin{table}[H]
%	\centering
%	\scriptsize
%	\caption[A demonstration of a table typeset in \LaTeX{}.]{An example of a table formatted with caption.}
%	
%	% Tune the following two values that are being multiplied by the variable \textwidth
%	% to control how large the scale of the table is, and how much is is squashed back 
%	% to the final size.
%	\resizebox{0.8\textwidth}{!}{
%		\begin{tabularx}{0.46\textwidth}{c|c|c|c|c|c}
%			\toprule
%			Some & Relevant & Fields & From & Your & Data\\
%			\midrule
%			0 & 0 & 0 & 0 & 0 & 0\\
%			1 & 1 & 1 & 1 & 1 & 1\\
%			2 & 2 & 2 & 2 & 2 & 2\\
%			\bottomrule
%		\end{tabularx}
%		\label{tbl:example_table}
%	}
%\end{table}


