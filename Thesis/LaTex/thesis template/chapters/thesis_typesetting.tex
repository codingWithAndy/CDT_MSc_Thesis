% !TEX root = ../thesis.tex
\chapter{Methodology}
	\label{chap:typesetting}
	
	\section{Tools}
	To create the web application and insights from the tweets, we required to use several tools. It is required that we develop a full-stack web application with a user UI, an area to input the user's judgements on the tweet, store the results using a database, and extract information from the tweets using NLP techniques. Several factors within the final application needed to be satisfied for the tools to be appropriate for use.
	
	\subsection{Programming Language}
	While many programming languages can handle creating a full-stack application and conducting ML, for example, Java, Php and JavaScript. We decided to use the Python language. We decided upon Python due to our familiarity with it over the other main languages and its versatility. We made this decision because Python can make full-stack applications with the use of additional libraries, as well as handle most NLP ML tasks using libraries like NLTK, SpaCy and Sci-Kit Learn.
	
	\subsection{Libraries}
	\subsubsection{Web Application}
	For creating the web application, there were two main libraries available. These were Django and Flask.
	
	Django is a high-level Python Web framework that encourages rapid development and clean, pragmatic design. Built by experienced developers, it takes care of much of the hassle of Web development, so you can focus on writing your app without needing to reinvent the wheel. It’s free and open source \cite{django}.
	
	\subsubsection{NLP Tasks}
	
	\subsection{IDE}
	
	\section{Software Development Life Cycle Methodology}
	
	
	\section{Data Set}
	
	\subsection{Data Capture Method}
	
	\subsection{Pre-Processing}
	
	
	