% !TEX root = ../thesis.tex
% The lstinline command can be used to insert monospace formatted code directly 
% inline within the documents main text. You can optionally specify a programming
% language to enable syntax highlighting. 
\lstinline|the_code|
\lstinline[language={the_language}]|the_code|

% The lstinputlisting command is used to insert an external file containing 
% code into the document formatted in the same manner as a figure or table. 
% All stand alone listings should have a label and caption. You can optionally
% specify a programming language to enable syntax highlighting. 
\lstinputlisting[label={lst:my_label_name}, caption={The caption.}]{the_file}
\lstinputlisting[language={the_language}, label={lst:the_label}, caption={The caption.}]{the_file}

% An example showing how Listing 3.1 is formatted in LaTeX code. 
% The C code is stored in its own file as C code, allowing it to be modified
% and prepared separately using a dedicated code IDE to ensure correctness and 
% proper formatting. 
\lstinputlisting[language={c}, label={lst:c_hello_world}, caption={An implementation of an important algorithm from our work.}]{./listings/hello_world.c}