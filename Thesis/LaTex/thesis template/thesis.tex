%% NOTE: The LaTeX format of this template is the authoritative version, but if you 
%% wish you may convert it to, e.g., Word or Google Docs format for editing. Recent 
%% versions of Word and LibreOffice can natively edit PDF files, as can Google Docs
%% (with minor formatting issues you will need to correct). Alternatively, you can use 
%% pandoc (and other free online conversion tools such as those listed here: 
%% https://softwarerecs.stackexchange.com/questions/1361/)

% Global template configuration (see custard.cls for details)
% Two-sided means the left and right margins are different sizes and they alternate 
% every page. If your document is printed to be book or spiral bound this allows for a 
% thick spine that does not eat into the space for your page content. 
% You can also add "draft" as an option if needed, which clearly marks the document 
% as a draft, hides the declaration, dedication, acknowledgements and appendix, 
% and forces one-sided mode to save space.
\documentclass[11pt, a4paper, twoside, openright]{custard}


% All imports, packages and configuration go in here. Your document should be 
% about content, so we abstract away the styling rules and tools we are using. 
% !TEX root = thesis.tex
%% Here you can specify new packages, commands and environments that you intend
%% to use. Using custom commands (for example, those for e.g., i.e., etc. below)
%% can make your document easier to write, read and more consistent.
\usepackage{float} % Adds the [H] option for forced figure placement
\usepackage[norefs,nocites]{refcheck} % Check and warn about unused labels
\usepackage[export]{adjustbox} % Used to add a frame around figures

% We are writing in British English
\usepackage[british]{babel}

% How many levels of sections/subsections etc to display in the Table of Contents
\setcounter{tocdepth}{1}

% Space out lines slightly - 1.5 spacing is overly severe, so we go for a 
% more visually pleasing 1.31
\linespread{1.31}

% Common shortcuts for consistency - this allows you to write, for example, 
% \eg in LaTeX instead of typing e.g., so that every single instance will be 
% formatted identically. If you later want to change one of these definitions, 
% all usages throughout the document will be updated.
\newcommand{\eg}{e.g.,\xspace}
\newcommand{\ie}{i.e.,\xspace}
\newcommand{\etc}{etc.\@\xspace}
\newcommand{\cf}{cf.\xspace}
\newcommand{\vs}{vs.\xspace}
\newcommand{\etal}{et al.\xspace}
\newcommand{\sd}{s.d.\xspace}
\newcommand{\elide}{[\,\ldots]\xspace}
\newcommand{\edots}{\,\ldots}

% Figure caption formatting
\usepackage[font=small,skip=1em]{caption}
\usepackage[labelformat=simple]{subfig}
\renewcommand{\thesubfigure}{\alph{subfigure})}

% Code listing formatting
\usepackage[final]{listings} % "final" option means show listings even in draft mode
\usepackage{lstautogobble}
\usepackage{sourcecodepro}
\pdfmapfile{=SourceCodePro.map}
\lstset{
	xleftmargin=0.5cm,frame=tlbr,framesep=4pt,framerule=0.5pt,
	language=,
	upquote=true,
	columns=fixed,
	tabsize=2,
	extendedchars=true,
	breaklines=true,
	numbers=left,
	numbersep=10pt,
	basicstyle=\ttfamily\scriptsize,
	numberstyle=\tiny,
	stringstyle=\ttfamily,
	captionpos=b,
	showstringspaces=false,
	autogobble=true
}

% Use IEEEtran citation style
\bibliographystyle{IEEEtran} 


\begin{document}
	
% The custom data for Swansea University and your degree.
% Your name and (for all except Doctoral theses), your student number
	\title{Rate My Tweet: Understanding Comparative Judgement in the Wild}
	\author{Andy Gray}
	\studentnumber{445348}
	\awardinginst{Swansea University}
	
% Comment / uncomment your degree type as needed.
	%\degree{Bachelor of Science} 
	\degree{Master of Science}
	%\degree{Doctor of Philosophy}
	
% Institution details and logo
	\department{Department of Computer Science}
	\university{Swansea University}
	\unilogo{swansea-logo.pdf}
	
% Hard code the date or allow the LaTeX compiler to fill it in 
% whenever you recompile the document.
	\date{30th September 2021}
	%\date{\today}
	
% Build the title and declaration pages, and pad the document so the text starts 
% on a right-hand book page. Page numbering is in roman numerals until the first 
% page of an actual chapter, which resets numbers starting from 1 at that point. 
	\frontmatter
	\maketitle
	\declaration
	\cleardoublepage
	
% Most significant books and theses have a brief foreword or dedication. 
% Shorter documents normally do not - remove / comment out if necessary.
	\ifdraftdoc\else
	\begin{vplace}[0.7]
		\begin{large}
			\begin{center}
				\textit{I would like to dedicate this work to my daughter and my family. Thank you for all of your support.}
			\end{center}
		\end{large}
	\end{vplace}
	\fi

% The abstract comes before the contents page. Note how each sentence is 
% written on a new line - this is an optional convention that can make it easier 
% to compare versions of your document using automated tools (e.g., diff).
	\begin{abstract}
		\vspace{-2em}
		\setcounter{page}{1}
		Marking and feedback is such an essential part of teaching and learning. For students to improve, they need to receive feedback. However, for the students to receive the feedback, the teachers need to mark it. Marking takes a considerable time for the teacher to complete and creates a significant cognitive load within the process. Therefore an alternative approach to marking called adaptive comparative judgement (ACJ) has been proposed in the educational space. ACJ has derived from the law of comparative judgment (LCJ), a pairwise method that compares and ranks items. While studies suggest that ACJ is highly reliable and accurate while making it quick for the teachers, alternative studies have questioned this claim suggesting that the process can bias the results through its adaptive nature. Additionally, studies have also found out that the ACJ can result in the overall marking process taking longer than a more traditional method of marking. At the same time, the current ACJ applications provide little resources in personalised feedback to individual students.
		
		Therefore, we have proposed a new ranking system that can rank the outcomes from the comparative judgement marking approach. The alternative ranking system was the Elo system. Additionally, aiming to reduce teachers cognitive load, reduce the time required to mark and ultimately provide personalised feedback to the user using NLP techniques. We experimented on Twitter tweets around the topic of Brexit to ask users what tweets they found funnier. The findings found that the Elo system is a suitable system to use for ranking the tweets outcomes. However, the NLP feedback process results provided good building blocks for future experiments that did not have a positive impact as desired.
		
		The code to this thesis project can be found here:
		\begin{center}
			\url{https://github.com/codingWithAndy/CDT_MSc_Thesis}
		\end{center}
	\end{abstract}
	
% A long form dedication (optional).
	\ifdraftdoc\else
	\begin{Acknowledgements}
	First, I would like to thank my partner and my daughter. Secondly, I would like to say a massive thank you to my supervisory team, Dr Alma Rahat, Professor Tom Crick and Dr Stephen Lindsay, for all of your advice. Additionally, I would like to thank Darren Wallace from CDSM for all your design ideas and input. 
	\end{Acknowledgements}
	\fi
	
% Build the table of contents page.
	\tableofcontents*
	 
% Optionally you can make a bank of known acronyms in acronyms.tex that you can call on throughout your document.
	%\input{acronyms} 
	
% For long documents like a Doctoral thesis you often include a list of tables and 
% figures that are used throughout your document. The command below uses a
% shortened version of each table and figure caption (specified by you - see examples 
% throughout) and enumerates all of them with their table or figure number. This 
% process is automatic - just uncomment the lines below to use it.
	%\newpage
	%\listoftables
	%\mtcaddchapter 
	%\newpage
	%\listoffigures
	%\mtcaddchapter

% If you use todo notes, you want to make sure you fix them all before final submission
	\ifnum\totvalue{todocounter}>0
		\listoftodos
	\fi

% Reset numeric page numbering from page 1
	\mainmatter

% Insert the source file for each of your chapters
	% !TEX root = ../thesis.tex
\chapter{Introduction}
	\label{chap:intro}
	We have set out to create a tool that can simulate a small scale comparative judgement experiment on what users think about tweets getting compared against each other. This experiment is in light of our stakeholder getting commissioned by the Welsh government to implement a comparative judgement system nationally for all schools in Wales. Comparative judgement is a technique that has been around for almost 100 years. However, while the process can improve results and reduce cognitive loads for teachers and markers, especially at the scale that the stakeholder's implementation will have to work at, it can still require many combinations to be marked and compared. For this experiment, we decided to use tweets based on Brexit.
	
	Therefore, we have created a tool that allows users to see a sub-sample of the combinations. Once the users have viewed the varieties, an overall ranking of the results will get created. Two methods got implemented, a more traditional comparative judgement method and an Elo style ranking.
	
	We then aimed to use NLP techniques to see any insights we could find within the tweets. We intended to extract information on the tweets to see if we could find patterns that would give us insights into what might have impacted the tweets final scores.
	
	The study got broken up into two parts. Part one was a web app to gather user's views on the tweets, and the second part was exploring NLP techniques within a Jupyter Notebook. With our aim to see if we can generate any feedback about the tweet.


	\section{Motivations}
		\label{sec:intro_motivation} 
	For the prior eight years, we have had involvement in some form of an educational environment. Seven of these years involve being a teacher within secondary and sixth form schools. While the focus of teaching is perceived to create lessons for students to learn and grow, we found more and more as the years went on that this wasn't the case. The focus was actually on providing reports about the students, which required data about the students from formal assessments. While having assessments to gauge the level that a student is at is an essential part of education. However, creating, marking, analysing and providing feedback for 30 students or more per class is a time-consuming task. Therefore, this assessment practice takes away the educators' time to do what is essential, creating meaningful lessons tailored for the students.
	
	Therefore, our motivation is to create a tool for educators that will empower them to allow technology to do what it is good at and focus on what they are good at, creating and delivering lessons. To shape future generations views.


	%\section{Objective}
	%	\label{sec:intro_objective} 


	\section{Existing Liturature}
	
	\section{New Insights}
	While the comparative judgement technique has many great features, we believe that the concept can still improve. We believe this is especially the case when the comparative judgment system gets expected to get done at a national scale. We believe this because the traditional method would expect all unique pairings to get compared. Additionally, the adaptive comparative judgement that most other systems have adopted still requires time and effort even when the number of individual student work is only around thirty. Therefore, it would be tough to do when needed to get scaled up to a national level. That is why we believe a different ranking system, like an Elo system, could replace the adaptive comparative judgement process and have a more crowed sourced approach. Therefore, reducing cognitive load and the time cost it would take for people to partake. 
	
	Furthermore, the current implementations do not provide any meaningful feedback to the students or educators about what makes a piece of work better than the other. Therefore, we think we can look into NLP techniques that can provide some form of feedback. To see if this can become something more meaningful and give some insights. Marking and giving feedback is a crucial role for all educators and the students receiving the feedback.


	\section{Contributions} 
		\label{sec:intro_contribs} 

		The main contributions of this work can be seen as follows:

		\begin{description}	

			\item[\(\bullet\) A \LaTeX{} thesis template]\hfill

			Modify this document by adding additional \TeX{} files for your top level content chapters. 

			\item[\(\bullet\) A typesetting guide for useful primitive elements]\hfill

			Use the building blocks within this template to typeset each part of your document.
			Aim to use simple and reusable elements to keep your \LaTeX{} code neat and to make your document consistently styled throughout.

			\item[\(\bullet\) A review of how to find and cite external resources]\hfill

		We review techniques and resources for finding and properly citing resources from the prior academic literature and from online resources.

		\end{description}
	
	\section{Results Overview}
	
	\section{Overview}  
	\label{sec:intro_overview} 
	 We will first look into the background, explaining the need education has for marking, allowing educators to rank students' work, and providing feedback to students to enable them to reflect and improve. We will then look into what comparative judgement is and its different iterations. Additionally, we look into different ranking systems, with both coming from the chess world but get currently implemented in all other scenarios, like e-Sports. We then look into what Natural Language Processing (NLP) is and some techniques to help achieve what we aim to achieve within our implementation. Then finally for this section will look at other applications that aim to implement comparative judgment within them. We will then look at our methodology, explaining the tools and design approaches we decided to use. We then look at the results we found and a discussion around these. We then finish with a conclusion and suggested further work for this project.
	% !TEX root = ../thesis.tex
\chapter{Lit Review}
	\label{chap:lit_review}
	Education and the sharing of knowledge is a powerful tool. In fact, in our opinion the most important skill anyone can have. As a famous quote said, "give a man a fish, and he will starve, but teach him to fish, and he won't be hungry anymore". However, it wasn't until 1918 that education, as most people in England and Wales have experienced, started to come into effect \cite{education1918}.
	
	Education over the years was very much about just giving the knowledge to the students from the teacher. It wasn't until 1988, under the Education Reforms Act 1988, that assessments got introduced. The introduction was through the introduction of the national curriculum in England and Wales \cite{education1988}.
	
	As the curriculum got rolled out, statutory assessments got introduced to education between 1991 and 1995. Key Stage 1 first, followed by Key Stages 2 and 3, respectively \cite{hutchison1994reliable, dillon2011becoming}. Only for the core subjects of English, Mathematics and Science had the assessments first introduced. The first assessments in Key Stage 1 were a range of cross-curricular tasks to be delivered in the classroom, known as standardised assessment tasks - hence the common acronym 'SATs'. However, the complexity of the use of these meant more formal assessments quickly replaced them \cite{hutchison1994reliable, dillon2011becoming}. The assessments in Key Stages 2 and 3 got developed using more traditional tests.
	
	To allow teachers to judge students' attainment, taking tests became the main assessment form in key stage 3. While assessments were the main form, educators were also able to assess their students with other means against the targets set for attainment within the national curriculum \cite{dillon2011becoming}. The teacher and assessment outcomes got used on a scale with key learning milestones expected at different ages. A key stage level indicated the result for the students progress. The model was used throughout the next few years until 2005 when the role of tests in KS1 got downgraded to just being an internal support tool to teachers, and in then 2008, the government decided to remove tests in KS3 \cite{dillon2011becoming}.
	
	This model continued, with minor adjustments to reflect the changing content of the National Curriculum, up to 2004. From 2005, the role of the tests got downplayed at Key Stage 1, with tests being used only internally to support teacher assessment judgements \cite{bbc_no_tests}. Further changes came in 2008 when the government announced that testing in Key Stage 3 was to get scrapped altogether \cite{bbc_tests_scrapped}.
	
	However, with a change of government party, the Conservative party taking power from the Labour party brought about new changes to how education's focuses and pedagogy methods would get conducted. In 2014 the system of attainment levels was removed, creating the educational shift of "Assessing without level" \cite{ass_without_lvls}. However, within schools, it was being referred to as 'life after levels'. Especially by our educational colleges and us at the time. Which was the follow up to the changes in the national curriculum in 2013 \cite{ass_without_lvls}. The changes within the national curriculum brought a greater focus on more traditional style GCSE academic subjects while reducing the focus on perceived technical labour style jobs. The new curriculum direction created more emphasis on the final exam outcomes at the stages of GCSE and A-Level.
	
	\section{The Purpose of Assessment, Marking and Feedback in Education}
	As we have established, assessments became a staple of the UK educational system in 1988. While the term assessments are not usually defined, the word 'assess' is typically associated with measuring, determining, evaluating, and judging \cite{wellington2007secondary}.
	
	While there can be multiple reasons why educators assess students, assessments aim to serve a purpose to both the teacher and the student in the process. These include: giving feedback to teachers and learners; providing motivation and encouragement; to boost the self-esteem of the pupils; a basis for communication; a method to evaluate a lesson/training method/scheme of work/ curriculum; to entertain \cite{wellington2007secondary}. Additionally, the assessment also creates other opportunities to rank students; a method to select and filter students, allocate students a particular pathway or educational direction, or as a way to discriminate or choose between students for a given set reason \cite{wellington2007secondary}.
	
	\subsection{Traditional Methods of Assessment and Feedback} % Should these be subsections?
	There are four main categories of assessment. These are diagnostic, formative, summative, and national assessments \cite{wellington2007secondary, dillon2011becoming}. However, it is essential to note that national assessments do not get used within everyday aspects of teaching and learning. This term is the name given to the critical exams like SATS, GCSE and ALevel exams taken nationally. Therefore we will focus on the other three main ones.
	
	Diagnostic assessment is what gets referred to as pre-testing \cite{wellington2007secondary}. Educators use this technique to get a base level of knowledge of the students they have inherited. This method is good for showing the progress of attainment over time by having an initial base test. Teachers can then show how well the students have progressed over time with their improvements over the term. This base assessment also provides the teacher with crucial information - the current ability of every student's knowledge. Through knowing this current level of knowledge, teachers can adapt the coming lessons and provide suitable differentiation and scaffolding within the lessons to allow each student to succeed as much as possible. However, we also experienced, within our time as an educator, the technique getting used to create baseline narratives. Teachers were using them to show that the student's knowledge wasn't at the expected level when inherited by the teacher at meetings or performance management reviews. Therefore, being used as a counter-act measure tool by the teacher, if they find themselves being accused of letting the students' performance slip, by trying to counter-act by implying the students were not at the required level in the first place.
	
	The second method, formative assessment, is also known as 'assessment for learning (AFL)' \cite{wellington2007secondary, dillon2011becoming}. This method has become one of the main tools for a teacher in terms of assessment and feedback. AFL allows the educator to assess the students' understanding of a topic on the fly during a lesson without a summative assessment. As a result, allowing the teacher to spend more or less time if the students do or don't get the topic, even if they planned more or less time for that topic. Therefore, ensuring that the teaching is not getting carried out for teaching sake. Thus, the emphasis is less on measurements and more on actual learning. AFL can involve using several techniques: teacher assessment - through in-class questions, marking books; to the students assessing their work called self-assessment, or peer assessment - where the students evaluate each other's work \cite{wellington2007secondary}.
	
	The third method is a summative assessment, also known as 'assessment of learning (AOL) \cite{wellington2007secondary}. This type of assessment happens at the end of a teaching unit or topic. It gets used to gain insights into what the students have learnt within the subject covered or the course. Its purpose is to give a student a mark, grade or ranking. Usually, this is the grade that is mainly focused on, as it is the metric that will impact the school the most in terms of league performance tables regarding GCSE and A-level results. From our experience, summative assessments are carried out regularly within schools. This assessment method tends to get used to getting a snapshot of the students of whit if a moment like, if they were to take the test now, what would they get? By seeing the results, educators can see if students need to attend intervention or if they are performing as expected or even better. With so much riding on these results, for schools and teachers performance management reviews, a lot of emphasis on put into trying to predict the final results for students. We have seen it put a lot of pressure on the teachers and the students and ultimately creates a very stressful environment, which is not the best environment for learning.
	
	
	
	
	\subsection{Why Traditional Traditional Marking and Feedback Methods are Effective}
	
	
	\subsection{The Negative Aspects of Traditional Marking and Feedback Methods}
	
	\section{Comparative Judgement}
	
	
	\subsection{What is Comparative Judgement} 
		Comparative judgement is a mathematical way to determine which observation item is better than the other item also being observed compared to each other. This method was first proposed in 1927 by Louis Leon Thurstone, a psychologist, under the term "the law of comparative judgement" \cite{thurstone1927psychophysical, thurstone1927law}. In modern-day terminology, it gets more aptly described as a model used to obtain measurements from any pairwise comparison process. Examples of such methods are comparing the perceived intensity of physical stimuli, such as the weights of objects, and comparing the extremity of an attitude expressed within statements, such as statements about capital punishment. The measurements represent how we perceive things rather than being measurements of actual physical properties. This kind of measurement is the focus of psychometrics and psychophysics. <wikipedia>
		
		In more technical terms, the law of comparative judgment is a mathematical representation of a discriminal process. This process involves a comparison between pairs of a collection of entities concerning multiple magnitudes of attributes. The model's theoretical basis is closely related to item response theory and the theory underlying the Rasch model. These methods are used in psychology and education to analyse data from questionnaires and tests.
		<wikipedia>
		
		While comparative judgement is a technique that has been around for almost 100 years, it wasn't until the early nineties that this technique got proposed for use within an educational setting. This first proposal was by Politt and Murry \cite{pollitt1996raters}, who conducted a study where they tested candidates on their English proficiency within Cambridge's CPE speaking exam. The judges watched 2-minute videos and judged which one out of a pair of videos they deemed better at the requested task in the exam. However, before this, in the ninety seventies and eighties, comparative judgement was presented as a more theoretical basis for educational assessments \cite{andrich1978rating}. 
		
		With the momentum of his findings, Politt then presented comparative judgement as a tool for exam boards to use to be able to compare the standards of A-Levels from the different exam boards, replacing the direct judgement of a script that was at the time currently being used \cite{newton2007paired}. In his papers titled, "Let's Stop Marking Exams" \cite{stop_marking_pollitt}, he presents a valid argument for using comparative judgement, with the advantages it brings over some traditional types of marking.
		
		Politt, in 2010, also presented a paper at the Association for Educational Assessment – Europe. It was about How to Assess Writing Reliably and Validly. Politt presented evidence of the extraordinarily high reliability achieved with Comparative Judgement in assessing primary school pupils' skill in first-language English writing \cite{pollitt2009abolishing}.
		
	\subsection{The Logic Behind Comparative Judgement and What it Aims to Do} % Should these be subsections?
		How comparative judgement works is to present two options to a marker. The marker then gets asked to pick which one of the two options they think is the better one. The marker will get presented with all possible combinations available, each time picking which one they think is the better one out of the two. An outputted score is then presented based on the method used. The original method, the Law of Comparative Judgement (LCJ), follows the formula:
		
		\begin{figure}[h]
			\includegraphics[width=8cm]{graphics/LCJ_formula.png}
			\caption{}
			\centering
		\end{figure}
	
		 $S_{i}$ is the psychological scale value of stimuli $i$
		%{\displaystyle x_{ij}} is the sigma corresponding with the proportion of occasions on which the magnitude of stimulus i is judged to exceed the magnitude of stimulus j
		%{\displaystyle \sigma _{i}} is the discriminal dispersion of a stimulus {\displaystyle R_{i}}
		%{\displaystyle r_{ij}} is the correlation between the discriminal deviations of stimuli i and j
		%The discriminal dispersion of a stimulus i is the dispersion of fluctuations of the discriminal process for a uniform repeated stimulus, denoted {\displaystyle R_{i}}, where {\displaystyle S_{i}} represents the mode of such values. Thurstone (1959, p. 20) used the term discriminal process to refer to the "psychological values of psychophysics"; that is, the values on a psychological continuum associated with a given stimulus.
		
		However, an alternative version derived from Louis Leon Thurstone, referred to as the "Pairwise Comparison" \cite{thurstone1927law}, will provide an output based on the difference between the quality values is equal to the log of the odds in respect to object-A will be object-B. This formula gets represented as: 
		$\displaystyle \mathrm {log\;odds} (A\ {\text{beats}}\ B\mid v_{a},v_{b})=v_{a}-v_{b} $.
		
		$\Pr\{X_{ji}=1\}={\frac {e^{{\delta _{j}}-{\delta _{i}}}}{1+e^{{\delta _{j}}-{\delta _{i}}}}}=\sigma (\delta _{j}-\delta _{i})$
		
		 .
		%\displaystyle \mathrm {log\;odds} (A\ {\text{beats}}\ B\mid v_{a},v_{b})=v_{a}-v_{b}}		


	\subsection{How effective is Comparative Judgement at Providing Feedback?} % Should these be subsections?
		
		

	\section{Related Work}
		\label{sec:google_fu}



	\subsection{Subsection all similar work}
		\label{sec:resources_bibtex}
	


	\subsection{Comparison of similar work}

	% !TEX root = ../thesis.tex
\chapter{Methodology}
	\label{chap:typesetting}
	
	\section{Tools}
	To create the web application and insights from the tweets, we required to use several tools. It is a requirement that we develop a full-stack web application with a user UI, an area to input the user's judgements on the tweet, store the results using a database, and extract information from the tweets using NLP techniques. Several factors within the final application needed to be satisfied for the tools to be appropriate for use.
	
	
	\subsection{Programming Language}
	While many programming languages can handle creating a full-stack application and conducting ML, for example, Java, Php and JavaScript. We decided to use the Python language. We decided upon Python due to our familiarity with it over the other main languages and its versatility. We made this decision because Python can make full-stack applications with the use of additional libraries, as well as handle most NLP ML tasks using libraries like NLTK, SpaCy, Sci-Kit Learn and TensorFlow.
	
	\subsection{Libraries}
	\subsubsection{Web Application}
	For creating the web application, there were two main libraries available. These were Django and Flask.
	
	Django is a high-level Python Web framework that encourages rapid development and clean, pragmatic design. Built by experienced developers, it takes care of much of the hassle of Web development, so you can focus on writing your app without needing to reinvent the wheel. It’s free and open source \cite{django}.
	
	While Flask is a small framework by most standards—small enough to be called a “micro- framework,” and small enough that once you become familiar with it, you will likely be able to read and understand all of its source code \cite{grinberg2018flask}. 
	
	Flask has three main dependencies. The routing, debugging, and Web Server Gateway Interface (WSGI) subsystems come from Werkzeug; the template support is provided by Jinja2; and the command-line integration comes from Click. These dependencies are all authored by Armin Ronacher, the author of Flask \cite{grinberg2018flask}. 
	
	Flask has no native support for accessing databases, validating web forms, authenti‐ cating users, or other high-level tasks. These and many other key services most web applications need are available through extensions that integrate with the core pack‐ ages. As a developer, you have the power to cherry-pick the extensions that work best for your project, or even write your own if you feel inclined to. This is in contrast with a larger framework, where most choices have been made for you and are hard or sometimes impossible to change \cite{grinberg2018flask}.
	
	%decision and justification
	After experimenting with the two frameworks, we decided upon Flask. Flask got decided upon because of the short time frame to put the project together. Additionally, the lightweight nature of the framework also played a fact as we believe that as this will be just an initial prototype, all the other requirements that Django requires would be unessential additionals to the project. Therefore, taking focus away from what we believe is the main focus. 
	
	\subsubsection{NLP Tasks}
	
	\subsection{IDE}
	
	\section{Software Development Life Cycle Methodology}
	
	
	\section{Data Set}
	
	\subsection{Data Capture Method}
	
	\subsection{Pre-Processing}
	
	
	
	% !TEX root = ../thesis.tex
\chapter{Results and Discussion}
\label{chap:results}

We will first look at the web application results based on the user's feedback, and then we will look into the insights and potential feedback the NLP process could provide the user. We then also look to review the overall process. 

We will compare the web application's results against the comparative judgment, Elo ranking, and the score we created for the tweets on Twitter. With the insights of the NLP for feedback to the user, we will look at what insights got made. Additionally, we will look at if any of the knowledge extracted generated provides any meaningful feedback to the user.



\section{Tweet Ranking Results} 
\label{sec:reaults_ranking}

\begin{figure}[h]
	\centering
	\includegraphics[width=7cm]{combination_heat_map.png}
	\caption{The web applicaitons generated results compared agaist each other.}
	\label{fig:combinations}
	
\end{figure}

\begin{figure}[h]
	\centering
	\includegraphics[width=7cm]{combination_win_heat_map.png}
	\caption{The web applicaitons generated results compared agaist each other.}
	\label{fig:combination_wins}
	
\end{figure}


While looking at fig \ref{fig:web_app_results}, we can see that the Elo and comparative judgement ranking generated very similar results. However, as we can see, the tweets coming in 6th, 7th and 8th a slight variation in the results.

\begin{figure}[h]
	\centering
	\includegraphics[width=10cm]{web_app_results.png}
	\caption{The web applicaitons generated results compared agaist each other.}
	\label{fig:web_app_results}
	
\end{figure}

\section{NLP Feedback and Insights}
\label{sec:reaults_NLP}

\section{Overall Results}
\label{sec:reaults_NLP}


	% !TEX root = ../thesis.tex
\chapter{Conclusions and Future Work}
\label{chap:conclusion}

In this document we have demonstrated the use of a \LaTeX{} thesis template which can produce a professional looking academic document. 


\section{Contributions} 
\label{sec:conclusion_contributions}

The main contributions of this work can be summarised as follows:
\begin{description}	

	\item[\(\bullet\) A \LaTeX{} thesis template]\hfill

Modify this document by adding additional top level content chapters.
These descriptions should take a more retrospective tone as you include summary of performance or viability. 

	\item[\(\bullet\) A typesetting guide for useful primitive elements]\hfill

Use the building blocks within this template to typeset each part of your document.
Aim to use simple and reusable elements to keep your document neat and consistently styled throughout.

	\item[\(\bullet\) A review of how to find and cite external resources]\hfill

We review techniques and resources for finding and properly citing resources from the prior academic literature and from online resources. 

\end{description}


\section{Future Work}
\label{sec:conclusion_future_work}

Future editions of this template may include additional references to Futurama.

% Add todo notes with \td{note 1} or \tdi{note 2}
\td{Add this yourself and submit a pull request?}


% Insert the bibliography using citations contained in the file citations.bib
	\bibintoc % Whether to list the bibliography in the Table of Contents (or: \nobibintoc)
	\bibliography{citations} 
	
% In the appendix you might include a full code listing for an implemented algorithm 
% that you showed a small chunk of in one of your chapters. If you have extra graphs 
% you might enumerate them within the appendix and use \label{name} and \cref{name} 
% to automatically insert the correct section locations when you talk about them in your 
% chapters. It is *not* necessary to include all of your implementation code as an 
% appendix; instead, focus on the important highlights so they do not get drowned out.
% Within appendix.tex you should use chapters as the top level section dividers.
	\ifdraftdoc\else
	\appendix
	\addappheadtotoc
	% !TEX root = ../thesis.tex
\chapter{Implementation of a Relevant Algorithm}
\label{app:implementation_algorithm}

% Code listings should live in a code file, not embedded directly into your LaTeX code!
\lstinputlisting[language=c, caption={An implementation of an important algorithm from our work.}]{./listings/hello_world.c}


\chapter{Supplementary Data}
\label{app:supplementary_data}

The results of large ablative studies can often take up a lot of space, even with neat visualisation and formatting.
Consider putting full results in an appendix chapter and showing excerpts of interesting results in your chapters with detailed analysis.
You can use labels and references to refer the reader here for the full data.

\chapter{Web App Pages}
\label{app:web_designs}

\begin{figure}[h]
	\centering
	\includegraphics[width=10cm]{Home_page.png}
	\caption{}
	
		
	\end{figure} 
\begin{figure}[h]
	\centering
	\includegraphics[width=10cm]{Compare.png}
	\caption{}
	
		
	\end{figure} 

\begin{figure}[h]
	\centering
	\includegraphics[width=10cm]{Results.png}
	\caption{}
	
		
	\end{figure} 

%\begin{figure}[h]
%	\centering
%	\includegraphics[width=10cm]{What_is_CJ.png.png}
%	\caption{}

		
%	\end{figure} 
\begin{figure}[h]
	\centering
	\includegraphics[width=10cm]{feedback.png}
	\caption{}
		
	\end{figure} 

%\begin{figure}[h]
%	\centering
%	\includegraphics[width=10cm]{login.png}
%	\caption{}
	
%\end{figure} 


%\begin{figure}[h]
%	\centering
%	\includegraphics[width=10cm]{signup.png}
%	\caption{}
	
%\end{figure} 

\chapter{Designs}
We will next look at the initial designs compared against the file outcome of the web app. We will also explain the decisions made and what changes we made, and why. 

In total, there are five different pages within the web app. The web app has a home page, facilitates the comparative judgment procedure, results, and feedback.

\subsection{Home Page}

\subsection{Comparison Page}

\subsection{What is Comparative Judgement Page}

\subsection{Results Page}

\subsection{Feedback Page}


\chapter{Risks}


\begin{center}
	\includegraphics[width=13cm]{risk_table.png}
	%\caption{A visual representation of the processes pipline.}...
\end{center}



\chapter{Schedule}
\begin{landscape}
	
	\begin{center}
		\item\includegraphics[width=21cm]{ganttchart.png}
	\end{center}
\end{landscape}

\chapter{Software Development Life Cycle Methodology}
Project management is crucial for any task that is about to be carried out, even more so for software development. As a famous Benjamin Franklin quote says, "Failing to plan is planning to fail" \cite{plan_to_fail}. With this in mind, we must decide on the suitable project planning method that compliments our initial software design. From the waterfall method to Rapid application development (RAD) or the more modern methods of agile development, there are many methods that we could choose. We will explain the different methods we could use and what would be best for our solution and intended development method.

The profession of the software developer has existed since the first computers, but the practices and methods for developing software have evolved over timer \cite{SDLC}. The approaches have developed over the years to adapt to the ever-changing landscape of software development. The methods, known as software development life cycles (SDLC), vary in approach but fundamentally share the same goal. The main aims of the SDLC are to break the development up into stages. However, what changes with different SDLC is how these stages get carried out. The different stages are planning, requirements, designing and prototyping, software development, testing, deployment, operations, and maintenance \cite{SDLC}.

The first stage, planning, involves resource allocation, capacity planning, project scheduling, cost estimation, and provisioning \cite{SDLC}. The primary outcome of this stage is to have an overall plan of what we have and what we will need to complete our goal within the constraints like costs and times allowed. The second stage, requirements, is where Subject Matter Experts (SMEs.) guide on what would be needed to carry out the stakeholders' requirements \cite{SDLC}. The third stage, design and prototyping, is where the software architects and developers begin to design the software. The outcome of this stage would be documentation on the intended design patterns and design wireframes of the intended final software. The fourth stage, development, is where the software starts to get made based on the decisions made in design and prototyping, following the chosen methodology. The outcome will be testable, tangible software. The fifth stage, testing, is considered the most crucial stage \cite{SDLC}. It is essential to do all the code quality checking, unit testing, integration testing, performance testing and security testing. The sixth but by no means the final stage is deployment. This stage is when the code is ready to be shipped to the client or uploaded to the required app stores. However, the final stage is operations and maintenance. This stage is about ensuring that the software is getting used as it should and that any bugs that did not initially get picked up in testing are correct and removed from the software. 

%\subsubsection{Waterfall Method}
The waterfall method is a model where each section needs to be completed before moving onto the next stage, like a waterfall flowing down. For example, before we can start analysing the requirements, we need to complete the planning stage. Following the seven critical stages of SDLC, one after the other.

Like all models, they have their advantages and disadvantages. Advantages that this model has is that it is easy to use and follow, and by the way it is all set up, every stage will get finished before the next stage starts. The waterfall method also allows for the project to be easily managed, resulting in easier documentation \cite{cscm01slidesl5}. However, some of the disadvantages are that it is not very useful if the requirements are not very clear at the beginning. Another disadvantage is that once we have moved to the next stage, it is tough to go back to a previous stage to make any changes which therefore creates higher risks to development and has less flexibility \cite{cscm01slidesl5}. The model is best when changes in the project are stable, and the project is small, with the project requirements are clearly defined.

%\subsubsection{RAD: Rapid Application Development}
The overall aim of RAD is to create software projects with higher quality and faster by gathering requirements through workshops or focus groups. Then prototyping the product and then using reiterative user testing of designs early. RAD is the best model for when we need something created quickly and have a pool of users available to test prototypes. However, this approach can be costlym \cite{cscm01slides}. 

%\subsubsection{Spiral Method}
The Spiral Model is an SDLC methodology that aids in choosing the optimal process model. It combines aspects of the incremental build model, waterfall model and prototyping model but is different by a set of six invariant characteristics \cite{spiralmodel}. The Spiral Model main focus is on risk awareness and management. The risk-driven approach of the spiral model ensures the team is highly flexible within its approach and highly aware of the challenges they can expect down the road. The spiral model shines when stakes are highest, and significant setbacks are not an option \cite{spiralmodel}.


%\subsubsection{Agile Development}
The Agile methodology is a process by which a team can manage a project, which gets achieved by breaking up the project into several stages. It required constant collaboration with stakeholders, which leads to continuous iterations of improvement. In essence, Agile development is not a set methodology more of a manifesto aiming to uncover better ways to develop software. "Individuals and interactions over processes and tools. Working software over comprehensive documentation. Customer collaboration over contract negotiation. Responding to change over following a plan \cite{agilemanifesto}."

%\subsubsection{Decided Method}
The project's requirements have features that lend themselves well to the waterfall methodology. However, we would like to have an element of agile methodology within the development due to the application intending to get created in a modular way. Using the waterfall method will allow us to have a clear plan and requirements of what is needed, but by using the agile method, we can rotate between the software development and testing stages.

\chapter{Testing}

The web application was the part of the implementation that required rigorous testing. The testing was because the web app was the bit that users would be interacting with the study. Therefore, we needed to ensure the app was to a high standard not to detract away from the users' experience and solely focus on the application purpose, which is to select which tweet they think is funnier. 

We conducted multiple in-house testing using an internal server's localhost to ensure that the app was suitable. Additionally, we allowed a small number of users to test out the application. Once we were happy with the feedback, the application's data got reset and published to potential users.


	\fi
	
\end{document}