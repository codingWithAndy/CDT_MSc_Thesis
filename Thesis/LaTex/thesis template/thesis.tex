%% NOTE: The LaTeX format of this template is the authoritative version, but if you 
%% wish you may convert it to, e.g., Word or Google Docs format for editing. Recent 
%% versions of Word and LibreOffice can natively edit PDF files, as can Google Docs
%% (with minor formatting issues you will need to correct). Alternatively, you can use 
%% pandoc (and other free online conversion tools such as those listed here: 
%% https://softwarerecs.stackexchange.com/questions/1361/)

% Global template configuration (see custard.cls for details)
% Two-sided means the left and right margins are different sizes and they alternate 
% every page. If your document is printed to be book or spiral bound this allows for a 
% thick spine that does not eat into the space for your page content. 
% You can also add "draft" as an option if needed, which clearly marks the document 
% as a draft, hides the declaration, dedication, acknowledgements and appendix, 
% and forces one-sided mode to save space.
\documentclass[11pt, a4paper, twoside, openright]{custard}


% All imports, packages and configuration go in here. Your document should be 
% about content, so we abstract away the styling rules and tools we are using. 
% !TEX root = thesis.tex
%% Here you can specify new packages, commands and environments that you intend
%% to use. Using custom commands (for example, those for e.g., i.e., etc. below)
%% can make your document easier to write, read and more consistent.
\usepackage{float} % Adds the [H] option for forced figure placement
\usepackage[norefs,nocites]{refcheck} % Check and warn about unused labels
\usepackage[export]{adjustbox} % Used to add a frame around figures

% We are writing in British English
\usepackage[british]{babel}

% How many levels of sections/subsections etc to display in the Table of Contents
\setcounter{tocdepth}{1}

% Space out lines slightly - 1.5 spacing is overly severe, so we go for a 
% more visually pleasing 1.31
\linespread{1.31}

% Common shortcuts for consistency - this allows you to write, for example, 
% \eg in LaTeX instead of typing e.g., so that every single instance will be 
% formatted identically. If you later want to change one of these definitions, 
% all usages throughout the document will be updated.
\newcommand{\eg}{e.g.,\xspace}
\newcommand{\ie}{i.e.,\xspace}
\newcommand{\etc}{etc.\@\xspace}
\newcommand{\cf}{cf.\xspace}
\newcommand{\vs}{vs.\xspace}
\newcommand{\etal}{et al.\xspace}
\newcommand{\sd}{s.d.\xspace}
\newcommand{\elide}{[\,\ldots]\xspace}
\newcommand{\edots}{\,\ldots}

% Figure caption formatting
\usepackage[font=small,skip=1em]{caption}
\usepackage[labelformat=simple]{subfig}
\renewcommand{\thesubfigure}{\alph{subfigure})}

% Code listing formatting
\usepackage[final]{listings} % "final" option means show listings even in draft mode
\usepackage{lstautogobble}
\usepackage{sourcecodepro}
\pdfmapfile{=SourceCodePro.map}
\lstset{
	xleftmargin=0.5cm,frame=tlbr,framesep=4pt,framerule=0.5pt,
	language=,
	upquote=true,
	columns=fixed,
	tabsize=2,
	extendedchars=true,
	breaklines=true,
	numbers=left,
	numbersep=10pt,
	basicstyle=\ttfamily\scriptsize,
	numberstyle=\tiny,
	stringstyle=\ttfamily,
	captionpos=b,
	showstringspaces=false,
	autogobble=true
}

% Use IEEEtran citation style
\bibliographystyle{IEEEtran} 


\begin{document}
	
% The custom data for Swansea University and your degree.
% Your name and (for all except Doctoral theses), your student number
	\title{Thesis Template}
	\author{A. N. Other}
	\studentnumber{1234567}
	\awardinginst{Swansea University}
	
% Comment / uncomment your degree type as needed.
	%\degree{Bachelor of Science} 
	\degree{Master of Science}
	%\degree{Doctor of Philosophy}
	
% Institution details and logo
	\department{Department of Computer Science}
	\university{Swansea University}
	\unilogo{swansea-logo.pdf}
	
% Hard code the date or allow the LaTeX compiler to fill it in 
% whenever you recompile the document.
	\date{30th September 2020}
	%\date{\today}
	
% Build the title and declaration pages, and pad the document so the text starts 
% on a right-hand book page. Page numbering is in roman numerals until the first 
% page of an actual chapter, which resets numbers starting from 1 at that point. 
	\frontmatter
	\maketitle
	\declaration
	\cleardoublepage
	
% Most significant books and theses have a brief foreword or dedication. 
% Shorter documents normally do not - remove / comment out if necessary.
	\ifdraftdoc\else
	\begin{vplace}[0.7]
		\begin{large}
			\begin{center}
				\textit{I would like to dedicate this work to\edots}
			\end{center}
		\end{large}
	\end{vplace}
	\fi

% The abstract comes before the contents page. Note how each sentence is 
% written on a new line - this is an optional convention that can make it easier 
% to compare versions of your document using automated tools (e.g., diff).
	\begin{abstract}
		\vspace{-2em}
		\setcounter{page}{1}
In your abstract you should aim to summarise the core contributions of your work in the context of the problem domain.
Start by outlining the domain and the problems posed within it.
Discuss how the methods you focus on approach the relevant problems.
You should end your abstract by concretely stating the tangible outputs and deliverables you have created in order to complete your work on this document, and whether those outputs represent and improvement or alternative approach to existing methods.

Your abstract should be a couple or so paragraphs long, and roughly approximate the order and flow you then use for structuring the main document.
If a viewer has read your abstract then they should already understand at a high level what it is you have created and delivered, and whether it is better than or comparable to existing methods.
If your project is driven by a research hypothesis then the reader should know what that is at a high level from this section.
Reading on, little should surprise the viewer.

For paper submission of your thesis you should physically sign your name and add the date for each of the above declaration statements (black ink preferred).
For digital submissions it is normally enough to simply type your name (see custard.cls), though you should sign and date them digitally using a touch or stylus input if at all possible.
	\end{abstract}
	
% A long form dedication (optional).
	\ifdraftdoc\else
	\begin{Acknowledgements}
This is an opportunity to acknowledge and thank those who have supported you throughout your studies.
Friends and colleagues who you have studied alongside, your families, and your mentors within the department are the usual suspects. 
	\end{Acknowledgements}
	\fi
	
% Build the table of contents page.
	\tableofcontents*
	 
% Optionally you can make a bank of known acronyms in acronyms.tex that you can call on throughout your document.
	%\input{acronyms} 
	
% For long documents like a Doctoral thesis you often include a list of tables and 
% figures that are used throughout your document. The command below uses a
% shortened version of each table and figure caption (specified by you - see examples 
% throughout) and enumerates all of them with their table or figure number. This 
% process is automatic - just uncomment the lines below to use it.
	%\newpage
	%\listoftables
	%\mtcaddchapter 
	%\newpage
	%\listoffigures
	%\mtcaddchapter

% If you use todo notes, you want to make sure you fix them all before final submission
	\ifnum\totvalue{todocounter}>0
		\listoftodos
	\fi

% Reset numeric page numbering from page 1
	\mainmatter

% Insert the source file for each of your chapters
	% !TEX root = ../thesis.tex
\chapter{Introduction}
	\label{chap:intro}
	We have set out to create a tool that can simulate a small scale comparative judgement experiment on what users think about tweets getting compared against each other. This experiment is in light of our stakeholder getting commissioned by the Welsh government to implement a comparative judgement system nationally for all schools in Wales. Comparative judgement is a technique that has been around for almost 100 years. However, while the process can improve results and reduce cognitive loads for teachers and markers, especially at the scale that the stakeholder's implementation will have to work at, it can still require many combinations to be marked and compared. For this experiment, we decided to use tweets based on Brexit.
	
	Therefore, we have created a tool that allows users to see a sub-sample of the combinations. Once the users have viewed the varieties, an overall ranking of the results will get created. Two methods got implemented, a more traditional comparative judgement method and an Elo style ranking.
	
	We then aimed to use NLP techniques to see any insights we could find within the tweets. We intended to extract information on the tweets to see if we could find patterns that would give us insights into what might have impacted the tweets final scores.
	
	The study got broken up into two parts. Part one was a web app to gather user's views on the tweets, and the second part was exploring NLP techniques within a Jupyter Notebook. With our aim to see if we can generate any feedback about the tweet.


	\section{Motivations}
		\label{sec:intro_motivation} 
	For the prior eight years, we have had involvement in some form of an educational environment. Seven of these years involve being a teacher within secondary and sixth form schools. While the focus of teaching is perceived to create lessons for students to learn and grow, we found more and more as the years went on that this wasn't the case. The focus was actually on providing reports about the students, which required data about the students from formal assessments. While having assessments to gauge the level that a student is at is an essential part of education. However, creating, marking, analysing and providing feedback for 30 students or more per class is a time-consuming task. Therefore, this assessment practice takes away the educators' time to do what is essential, creating meaningful lessons tailored for the students.
	
	Therefore, our motivation is to create a tool for educators that will empower them to allow technology to do what it is good at and focus on what they are good at, creating and delivering lessons. To shape future generations views.


	%\section{Objective}
	%	\label{sec:intro_objective} 


	\section{Existing Liturature}
	
	\section{New Insights}
	While the comparative judgement technique has many great features, we believe that the concept can still improve. We believe this is especially the case when the comparative judgment system gets expected to get done at a national scale. We believe this because the traditional method would expect all unique pairings to get compared. Additionally, the adaptive comparative judgement that most other systems have adopted still requires time and effort even when the number of individual student work is only around thirty. Therefore, it would be tough to do when needed to get scaled up to a national level. That is why we believe a different ranking system, like an Elo system, could replace the adaptive comparative judgement process and have a more crowed sourced approach. Therefore, reducing cognitive load and the time cost it would take for people to partake. 
	
	Furthermore, the current implementations do not provide any meaningful feedback to the students or educators about what makes a piece of work better than the other. Therefore, we think we can look into NLP techniques that can provide some form of feedback. To see if this can become something more meaningful and give some insights. Marking and giving feedback is a crucial role for all educators and the students receiving the feedback.


	\section{Contributions} 
		\label{sec:intro_contribs} 

		The main contributions of this work can be seen as follows:

		\begin{description}	

			\item[\(\bullet\) A \LaTeX{} thesis template]\hfill

			Modify this document by adding additional \TeX{} files for your top level content chapters. 

			\item[\(\bullet\) A typesetting guide for useful primitive elements]\hfill

			Use the building blocks within this template to typeset each part of your document.
			Aim to use simple and reusable elements to keep your \LaTeX{} code neat and to make your document consistently styled throughout.

			\item[\(\bullet\) A review of how to find and cite external resources]\hfill

		We review techniques and resources for finding and properly citing resources from the prior academic literature and from online resources.

		\end{description}
	
	\section{Results Overview}
	
	\section{Overview}  
	\label{sec:intro_overview} 
	 We will first look into the background, explaining the need education has for marking, allowing educators to rank students' work, and providing feedback to students to enable them to reflect and improve. We will then look into what comparative judgement is and its different iterations. Additionally, we look into different ranking systems, with both coming from the chess world but get currently implemented in all other scenarios, like e-Sports. We then look into what Natural Language Processing (NLP) is and some techniques to help achieve what we aim to achieve within our implementation. Then finally for this section will look at other applications that aim to implement comparative judgment within them. We will then look at our methodology, explaining the tools and design approaches we decided to use. We then look at the results we found and a discussion around these. We then finish with a conclusion and suggested further work for this project.
	% !TEX root = ../thesis.tex
\chapter{Lit Review}
	\label{chap:lit_review}
	
	
	\section{What is Comparative Judgement}
		Comparative judgement is a mathematical way to determine which observation item is better than the other item also being observed compared to each other. This method first got proposed in 1927 by Louis Leon Thurstone, a psychologist, under the term "the law of comparative judgement" \cite{thurstone1927psychophysical, thurstone1927law}. While comparative judgement is a technique that has been around for almost 100 years, it wasn't until the early nineties that this technique got proposed for use within an educational setting. This first proposal was by Politt and Murry \cite{pollitt1996raters}, who conducted a study where they tested candidates on their English proficiency within Cambridge's CPE speaking exam. The judges watched 2-minute videos and judged which one out of a pair of videos they deemed better at the requested task in the exam. However, before this, in the ninety seventies and eighties, comparative judgement was presented as a more theoretical basis for educational assessments \cite{andrich1978rating}. 
		
		With the momentum of his findings, Politt then presented comparative judgement as a tool for exam boards to use to be able to compare the standards of A-Levels from the different exam boards, replacing the direct judgement of a script that was at the time currently being used \cite{newton2007paired}. In his papers titled, "Let's Stop Marking Exams" \cite{stop_marking_pollitt}, he presents a valid argument for using comparative judgement, with the advantages it brings over some traditional types of marking.
		
		How comparative judgement works is to present two options to a marker. The marker then gets asked to pick which one of the two options they think is the better one. The marker will get presented with all possible combinations available, each time picking which one they think is the better one out of the two. An outputted score is then presented based on the method used. The original method, the Law of Comparative Judgement (LCJ), follows the formula:
		
		\begin{figure}[h]
			\includegraphics[width=8cm]{graphics/LCJ_formula.png}
			\centering
		\end{figure}
	
		 $S_{i}$ is the psychological scale value of stimuli $i$
		%{\displaystyle x_{ij}} is the sigma corresponding with the proportion of occasions on which the magnitude of stimulus i is judged to exceed the magnitude of stimulus j
		%{\displaystyle \sigma _{i}} is the discriminal dispersion of a stimulus {\displaystyle R_{i}}
		%{\displaystyle r_{ij}} is the correlation between the discriminal deviations of stimuli i and j
		%The discriminal dispersion of a stimulus i is the dispersion of fluctuations of the discriminal process for a uniform repeated stimulus, denoted {\displaystyle R_{i}}, where {\displaystyle S_{i}} represents the mode of such values. Thurstone (1959, p. 20) used the term discriminal process to refer to the "psychological values of psychophysics"; that is, the values on a psychological continuum associated with a given stimulus.
		
		However, an alternative version derived from Louis Leon Thurstone, referred to as the "Pairwise Comparison" \cite{thurstone1927law}, will provide an output based on the difference between the quality values is equal to the log of the odds in respect to object-A will be object-B. This formula gets represented as: 
		\displaystyle \mathrm {log\;odds} (A\ {\text{beats}}\ B\mid v_{a},v_{b})=v_{a}-v_{b} .
		
		\Pr\{X_{ji}=1\}={\frac {e^{{\delta _{j}}-{\delta _{i}}}}{1+e^{{\delta _{j}}-{\delta _{i}}}}}=\sigma (\delta _{j}-\delta _{i})
		
		 .
		%\displaystyle \mathrm {log\;odds} (A\ {\text{beats}}\ B\mid v_{a},v_{b})=v_{a}-v_{b}}		


	\section{Tunnel your internet connection via the university internet}

When you are working from outside of the University, connecting to an on-campus machine via remote desktop (RemoteDesktopProtocol, TeamViewer, \etc) or via port forwarding (ssh, ssh tunnel, ect) can allow you to access papers that would otherwise be behind a paywall. 
		
If you do not have individual access to a machine that is exposed for SSH on the University network you can always use the computers in Linux Lab CF204\footnote{One caveat of using computer lab machines for remote tunnelling is that a environmentally conscious student who has worked late in the computer lab might choose to switch off the machine you were using\edots} for the purpose of setting up an SSH port tunnel to proxy your internet through.
These machines have fixed IPv4 addresses and respond to SSH using your student account credentials.
While in use your internet will be routed\footnote{Painfully slowly.} to the University and then out to the internet, granting you transparent access to journals without a paywall.


	\section{Practice your Google Fu}
		\label{sec:google_fu}

%The internet is big \cite{sizeofinternet}.
%Knowing how to phrase a question to a search engine is therefore an invaluable skill.
If the request is simple enough, even a poorly structured query will likely return usable results.
For more difficult to find resources you can leverage the language of the search engine to gather relevant papers and resources for your research more efficiently. 
		
		% An example of how to centre a passage of text, control local font size, 
		% and create a properly formatted and clickable URL.
		\begin{center}
		{\small \url{https://www.gwern.net/Search}}
		\end{center}
		
``Internet Search Tips'' \cite{gwern} provides an excellent review of methods and tips for scouring the internet for hard to find resources.
You will also be less likely to get caught behind journal paywalls when working remotely without a tunnel as your queries can be made to look for raw PDFs that are often released by the authors directly.


	\section{Organising your citations in Bib\TeX{}}
		\label{sec:resources_bibtex}
	
Bib\TeX{} is a language for specifying resource citations.
Every time you access and read an academic paper, take code from an online repository, or source the media such as images from existing works, you should create a Bib\TeX{} entry in a file that you keep throughout your research.
Software such as Mendeley \cite{mendeley} can help automate the process of building your Bib\TeX{} library of citations. 
		
		\lstinputlisting[label={lst:bibtex}, caption={An example Bib\TeX{} entry for an academic paper published in conference proceedings \cite{kaj86}.}]{./listings/example_bibtex.bib}
		
The Bib\TeX{} code listing above (\cref{lst:bibtex}) shows an example of how to cite an academic paper; in this case one of the central papers in Computer Graphics research.
The key \textbf{kaj86} is an arbitrary name chosen as a meaningful identifier for the resource.
In the document text we can call on this resource as an inline citation using the \LaTeX{} command \lstinline|\cite{kaj86}|, which produces \cite{kaj86} at the location it is called.
As long as a citation has been used at least once somewhere within the document then a formatted full citation will be created in the bibliography at the end of the document with the same citation number that is shown inline.
		
It is considerably easier to be disciplined in methodically taking note of the resources you access and make use of as you access them than it is to try and hunt them all down again at the time you need to write about them in your document.
Invest time in being organised and consistent upfront and it will be easier when you come to write up.


	\section{Properly using and formatting citations within the text}

Usually you would not put the URL of the resource you are citing directly in the text like is done previously in \cref{sec:google_fu}.
The citation for the resource \cite{gwern} is sufficient to reference it within the text given that full details of its location are then kept neatly within the bibliography at the end of the document.

In normal usage the purpose of a citation is not to direct the reader away from your thesis, but to justify and back up assertions you are making about the state of the domain.
If a reader questions your assertions then they can follow the rabbit hole of papers which will likely also make and justify assertions with even earlier papers from the literature. 

In the above case the intention is for the reader of this template to actually go to that resource and read what it has to say directly.
The link is therefore shown clearly within the main text to indicate that the reader should visit it.
This as opposed to wanting the reader to purely acknowledge that the facts which are within the resource legitimise the points made in this document, in which case a simple inline citation is the best way to back up your assertions.
\Cref{sec:typesetting_figures_citation} specifically touches on the best practice for how to cite images which you are importing from existing work. 

	% !TEX root = ../thesis.tex
\chapter{Methodology}
	\label{chap:typesetting}
	
	\section{Tools}
	To create the web application and insights from the tweets, we required to use several tools. It is required that we develop a full-stack web application with a user UI, an area to input the user's judgements on the tweet, store the results using a database, and extract information from the tweets using NLP techniques. Several factors within the final application needed to be satisfied for the tools to be appropriate for use.
	
	\subsection{Programming Language}
	While many programming languages can handle creating a full-stack application and conducting ML, for example, Java, Php and JavaScript. We decided to use the Python language. We decided upon Python due to our familiarity with it over the other main languages and its versatility. We made this decision because Python can make full-stack applications with the use of additional libraries, as well as handle most NLP ML tasks using libraries like NLTK, SpaCy and Sci-Kit Learn.
	
	\subsection{Libraries}
	\subsubsection{Web Application}
	For creating the web application, there were two main libraries available. These were Django and Flask.
	
	Django is a high-level Python Web framework that encourages rapid development and clean, pragmatic design. Built by experienced developers, it takes care of much of the hassle of Web development, so you can focus on writing your app without needing to reinvent the wheel. It’s free and open source \cite{django}.
	
	\subsubsection{NLP Tasks}
	
	\subsection{IDE}
	
	\section{Software Development Life Cycle Methodology}
	
	
	\section{Data Set}
	
	\subsection{Data Capture Method}
	
	\subsection{Pre-Processing}
	
	
	
	% !TEX root = ../thesis.tex
\chapter{Conclusions and Future Work}
\label{chap:conclusion}

In this document we have demonstrated the use of a \LaTeX{} thesis template which can produce a professional looking academic document. 


\section{Contributions} 
\label{sec:conclusion_contributions}

The main contributions of this work can be summarised as follows:
\begin{description}	

	\item[\(\bullet\) A \LaTeX{} thesis template]\hfill

Modify this document by adding additional top level content chapters.
These descriptions should take a more retrospective tone as you include summary of performance or viability. 

	\item[\(\bullet\) A typesetting guide for useful primitive elements]\hfill

Use the building blocks within this template to typeset each part of your document.
Aim to use simple and reusable elements to keep your document neat and consistently styled throughout.

	\item[\(\bullet\) A review of how to find and cite external resources]\hfill

We review techniques and resources for finding and properly citing resources from the prior academic literature and from online resources. 

\end{description}


\section{Future Work}
\label{sec:conclusion_future_work}

Future editions of this template may include additional references to Futurama.

% Add todo notes with \td{note 1} or \tdi{note 2}
\td{Add this yourself and submit a pull request?}


% Insert the bibliography using citations contained in the file citations.bib
	\bibintoc % Whether to list the bibliography in the Table of Contents (or: \nobibintoc)
	\bibliography{citations} 
	
% In the appendix you might include a full code listing for an implemented algorithm 
% that you showed a small chunk of in one of your chapters. If you have extra graphs 
% you might enumerate them within the appendix and use \label{name} and \cref{name} 
% to automatically insert the correct section locations when you talk about them in your 
% chapters. It is *not* necessary to include all of your implementation code as an 
% appendix; instead, focus on the important highlights so they do not get drowned out.
% Within appendix.tex you should use chapters as the top level section dividers.
	\ifdraftdoc\else
	\appendix
	\addappheadtotoc
	% !TEX root = ../thesis.tex
\chapter{Implementation of a Relevant Algorithm}
\label{app:implementation_algorithm}

% Code listings should live in a code file, not embedded directly into your LaTeX code!
\lstinputlisting[language=c, caption={An implementation of an important algorithm from our work.}]{./listings/hello_world.c}


\chapter{Supplementary Data}
\label{app:supplementary_data}

The results of large ablative studies can often take up a lot of space, even with neat visualisation and formatting.
Consider putting full results in an appendix chapter and showing excerpts of interesting results in your chapters with detailed analysis.
You can use labels and references to refer the reader here for the full data.

\chapter{Web App Pages}
\label{app:web_designs}

\begin{figure}[h]
	\centering
	\includegraphics[width=10cm]{Home_page.png}
	\caption{}
	
		
	\end{figure} 
\begin{figure}[h]
	\centering
	\includegraphics[width=10cm]{Compare.png}
	\caption{}
	
		
	\end{figure} 

\begin{figure}[h]
	\centering
	\includegraphics[width=10cm]{Results.png}
	\caption{}
	
		
	\end{figure} 

%\begin{figure}[h]
%	\centering
%	\includegraphics[width=10cm]{What_is_CJ.png.png}
%	\caption{}

		
%	\end{figure} 
\begin{figure}[h]
	\centering
	\includegraphics[width=10cm]{feedback.png}
	\caption{}
		
	\end{figure} 

%\begin{figure}[h]
%	\centering
%	\includegraphics[width=10cm]{login.png}
%	\caption{}
	
%\end{figure} 


%\begin{figure}[h]
%	\centering
%	\includegraphics[width=10cm]{signup.png}
%	\caption{}
	
%\end{figure} 

\chapter{Designs}
We will next look at the initial designs compared against the file outcome of the web app. We will also explain the decisions made and what changes we made, and why. 

In total, there are five different pages within the web app. The web app has a home page, facilitates the comparative judgment procedure, results, and feedback.

\subsection{Home Page}

\subsection{Comparison Page}

\subsection{What is Comparative Judgement Page}

\subsection{Results Page}

\subsection{Feedback Page}


\chapter{Risks}


\begin{center}
	\includegraphics[width=13cm]{risk_table.png}
	%\caption{A visual representation of the processes pipline.}...
\end{center}



\chapter{Schedule}
\begin{landscape}
	
	\begin{center}
		\item\includegraphics[width=21cm]{ganttchart.png}
	\end{center}
\end{landscape}

\chapter{Software Development Life Cycle Methodology}
Project management is crucial for any task that is about to be carried out, even more so for software development. As a famous Benjamin Franklin quote says, "Failing to plan is planning to fail" \cite{plan_to_fail}. With this in mind, we must decide on the suitable project planning method that compliments our initial software design. From the waterfall method to Rapid application development (RAD) or the more modern methods of agile development, there are many methods that we could choose. We will explain the different methods we could use and what would be best for our solution and intended development method.

The profession of the software developer has existed since the first computers, but the practices and methods for developing software have evolved over timer \cite{SDLC}. The approaches have developed over the years to adapt to the ever-changing landscape of software development. The methods, known as software development life cycles (SDLC), vary in approach but fundamentally share the same goal. The main aims of the SDLC are to break the development up into stages. However, what changes with different SDLC is how these stages get carried out. The different stages are planning, requirements, designing and prototyping, software development, testing, deployment, operations, and maintenance \cite{SDLC}.

The first stage, planning, involves resource allocation, capacity planning, project scheduling, cost estimation, and provisioning \cite{SDLC}. The primary outcome of this stage is to have an overall plan of what we have and what we will need to complete our goal within the constraints like costs and times allowed. The second stage, requirements, is where Subject Matter Experts (SMEs.) guide on what would be needed to carry out the stakeholders' requirements \cite{SDLC}. The third stage, design and prototyping, is where the software architects and developers begin to design the software. The outcome of this stage would be documentation on the intended design patterns and design wireframes of the intended final software. The fourth stage, development, is where the software starts to get made based on the decisions made in design and prototyping, following the chosen methodology. The outcome will be testable, tangible software. The fifth stage, testing, is considered the most crucial stage \cite{SDLC}. It is essential to do all the code quality checking, unit testing, integration testing, performance testing and security testing. The sixth but by no means the final stage is deployment. This stage is when the code is ready to be shipped to the client or uploaded to the required app stores. However, the final stage is operations and maintenance. This stage is about ensuring that the software is getting used as it should and that any bugs that did not initially get picked up in testing are correct and removed from the software. 

%\subsubsection{Waterfall Method}
The waterfall method is a model where each section needs to be completed before moving onto the next stage, like a waterfall flowing down. For example, before we can start analysing the requirements, we need to complete the planning stage. Following the seven critical stages of SDLC, one after the other.

Like all models, they have their advantages and disadvantages. Advantages that this model has is that it is easy to use and follow, and by the way it is all set up, every stage will get finished before the next stage starts. The waterfall method also allows for the project to be easily managed, resulting in easier documentation \cite{cscm01slidesl5}. However, some of the disadvantages are that it is not very useful if the requirements are not very clear at the beginning. Another disadvantage is that once we have moved to the next stage, it is tough to go back to a previous stage to make any changes which therefore creates higher risks to development and has less flexibility \cite{cscm01slidesl5}. The model is best when changes in the project are stable, and the project is small, with the project requirements are clearly defined.

%\subsubsection{RAD: Rapid Application Development}
The overall aim of RAD is to create software projects with higher quality and faster by gathering requirements through workshops or focus groups. Then prototyping the product and then using reiterative user testing of designs early. RAD is the best model for when we need something created quickly and have a pool of users available to test prototypes. However, this approach can be costlym \cite{cscm01slides}. 

%\subsubsection{Spiral Method}
The Spiral Model is an SDLC methodology that aids in choosing the optimal process model. It combines aspects of the incremental build model, waterfall model and prototyping model but is different by a set of six invariant characteristics \cite{spiralmodel}. The Spiral Model main focus is on risk awareness and management. The risk-driven approach of the spiral model ensures the team is highly flexible within its approach and highly aware of the challenges they can expect down the road. The spiral model shines when stakes are highest, and significant setbacks are not an option \cite{spiralmodel}.


%\subsubsection{Agile Development}
The Agile methodology is a process by which a team can manage a project, which gets achieved by breaking up the project into several stages. It required constant collaboration with stakeholders, which leads to continuous iterations of improvement. In essence, Agile development is not a set methodology more of a manifesto aiming to uncover better ways to develop software. "Individuals and interactions over processes and tools. Working software over comprehensive documentation. Customer collaboration over contract negotiation. Responding to change over following a plan \cite{agilemanifesto}."

%\subsubsection{Decided Method}
The project's requirements have features that lend themselves well to the waterfall methodology. However, we would like to have an element of agile methodology within the development due to the application intending to get created in a modular way. Using the waterfall method will allow us to have a clear plan and requirements of what is needed, but by using the agile method, we can rotate between the software development and testing stages.

\chapter{Testing}

The web application was the part of the implementation that required rigorous testing. The testing was because the web app was the bit that users would be interacting with the study. Therefore, we needed to ensure the app was to a high standard not to detract away from the users' experience and solely focus on the application purpose, which is to select which tweet they think is funnier. 

We conducted multiple in-house testing using an internal server's localhost to ensure that the app was suitable. Additionally, we allowed a small number of users to test out the application. Once we were happy with the feedback, the application's data got reset and published to potential users.


	\fi
	
\end{document}